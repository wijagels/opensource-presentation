\section{Analysis}\frame{\sectionpage}

%first slide
\begin{frame}{Stallman's Argument: Basis}
\begin{minipage}{0.5\textwidth}
\begin{itemize}
    \item A deontological standpoint
    \item Stallman as an ethical essentialist
    \begin{itemize}
      \item proprietary software
      \item restricted data formats
      \item internet services 
      \item surveillance 
    \end{itemize}
   \begin{itemize}
        \item ``always bring up [free software] as an ethical issue''
          (Stallman, 2011, para. 63)
      \end{itemize}  
  \end{itemize}
\end{minipage} %\hfill
\begin{minipage}{0.45\textwidth}
  \includegraphics[width = 0.75\textwidth]{kant.jpg}
\end{minipage}
\end{frame}


\begin{frame}{Stallman’s Argument: Logos}
\begin{itemize}
  \item Deductive reasoning
    \begin{itemize}
      \item tobacco and proprietary software comparison (Stallman, 2011, para. 55)
      \end{itemize}
      \item Contradictory premises
        \begin{itemize}
          \item dismissing economics of free digital society (para. 34)
          \item later addressing economics of digital media (para. 109)
        \end{itemize}
        \end{itemize}
\end{frame}

\begin{frame}{Stallman’s Argument: Pathos}
\begin{itemize}
  \item Use of strong characterizations
    \begin{itemize}
      \item ``Computers are Stalin’s dream'' (2011, para. 3) 
      \item All DRM should be illegal (para. 30) 
    \end{itemize}
  \item Strong appeals to tradition
    \begin{itemize}
      \item values derived from a non-digital society
      \item Amazon Kindle (para. 98)
    \end{itemize}    
  \item Calls Amazon Kindle (para. 98)
    \begin{itemize}
      \item an immediate end to digital surveillance
      \item “you can’t wait until there is another dictator” (para. 13) 
     \end{itemize}
\end{itemize}
\end{frame}


        